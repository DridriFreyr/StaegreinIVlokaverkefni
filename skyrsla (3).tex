%%%%%%%%%%%%%%%%%%%%%%%%%%%%%%%%%%%%%%%%%%%%%%%%%%%%%%%%%%%%%%%%
\documentclass[11pt,a4paper,titlepage]{article}
\usepackage[utf8]{inputenc} 
\usepackage[english,icelandic]{babel}
\usepackage{t1enc}
\usepackage{hyperref}
\usepackage{amsmath}
\usepackage{amsfonts}
\usepackage{amssymb}
\usepackage{mdwlist} 
\usepackage{graphicx}
\usepackage{listings}
\usepackage[numbered, framed]{mcode} 
\usepackage{color}
\definecolor{mygreen}{RGB}{30,165,13}
  \parindent 0pt
  \lstset{language=matlab}
  \lstset{inputencoding=latin1}

%% Til a� breyta st�r� og sta�setningu textans � s��unni
% \voffset=-1.0in
% \hoffset=-0.3in
% \textwidth=6in
% \textheight=10.2in

% Stillingar fyrir mcode
\lstset{frame=tb, language=Matlab, aboveskip=3mm, belowskip=3mm, 
showstringspaces=false, 
columns=flexible, 
basicstyle={\small\ttfamily}, 
numberstyle=\tiny\color{gray}, 
keywordstyle=\color{blue}, 
commentstyle=\color{mygreen}, 
stringstyle=\color{mauve}, 
breaklines=true, 
breakatwhitespace=true 
tabsize=3 } 

% Eigin skipanir

\newcommand{\R}{{\Bbb R}}  

\title{Töluleg greining (ST05G vor 2015)\\
Heimaverkefni I}
\date{\today}
\author{Gunna J�nsd�ttir og J�n Gunnuson}

\begin{document}


\section{Varmajafnvægi}
% Undirkafli
\subsection{}
Skrifa \verb|matlab|-forrit sem finnur nálgunarlausn á Sturm-Liouville-verkefninu (1.1) með Galerkin-aðferð og þúfugrunnföllum, þar sem gert er ráð fyrir að $p$ megi vera ósamfellt og að hægri hliðin megi innnihalda punktuppsprettu.

\textbf{Lausn:} 
Forritið er svona útlítandi


\lstinputlisting{Galerkinadferd.m}




\subsection{}
Rökstyðjið að aðferð Galerkins virki með ósamfellt $p$ eins og lýst er hér að framan. Það eina sem þarf að gera að sýna að hlutheildunin sem liggur t il grundvallar aðferðinni virki í þessu tilfelli þegar $p$ er ósamfellt eins og lýst er hér að framan. 
\par

\textbf{Lausn:} \par

Við getum sýnt fram á virkni Galerkins aðferðarinnar fyrir ósamfellt $p$ með því að skipta bilinu sem heilda á yfir, upp í búta.\par
Ef við ætlum okkur að beita aðferðinni á bili [a,b] þar sem fall $p$ er ósamfellt í endanlega mörgum innri-punktum $s_{i}$ á bilinu, nægir okkur að beita aðferð Galarkins á búta\par \begin{center}[a,$s_{i}$], [$s_{i}$,$s_{i+1}$],$\cdot \cdot \cdot$, [$s_{n}$,b].\end{center}
Ljóst er að þegar lagaðar eru saman lausnir fyrir hvern bút, þá munu liðir háðir p($s_{i}$) styttast út hver á móti örðum.\par
Við getum því notfært okkur þetta og lagt saman heildin fyrir hvert hlutbil til þess að fá nálgunina fyrir allt bilið.

Skoðum tilfellið þar sem $p$ er ósamfellt í einum punkti s á bili [a,b].
Heildum þá yfir búta [a,s] og [s,b] og fáum:\par
\begin{center}
	$\int_{a}^{s}-(p(x)*u{(x)}')dx + \int_{a}^{s}q(x)u(x)dx= \int_{a}^{s}f(x)$ \par
	og $\int_{s}^{b}-(p(x)*u{(x)}')dx + \int_{s}^{b}q(x)u(x)dx= \int_{s}^{b}f(x)$
\end{center}
Ef við leggjum svo heildin saman fáum við:\par
\begin{center}
	$(-p(s)u{(s)}'+p(a)u{(a)}')+(-p(b)u{(b)}'+p(s)u{(s)}')+ \int_{a}^{s}q(x)u(x)dx+ \int_{s}^{b}q(x)u(x)dx =\int_{a}^{s}f(x)+\int_{s}^{b}f(x)$
\end{center}
Sem gefur svo: 
\begin{center}
	$\int_{a}^{b}-(p(x)*u{(x)}')dx + \int_{a}^{b}q(x)u(x)dx= \int_{a}^{b}f(x)$
\end{center} \par
Það er því ljóst að punktur $p(s)$ hefur ekki áhrif á niðurstöðun og því virkar Galerkin aðferðin fyrir ósamfellt $p$. 
En til þess að svo megi vera þarf að gilda að þar sem $p(s_{i})$ er ósamfellt, þá þarf $p(s_{i})*u'(s_{i})$ að vera samfellt og $u'(s_{i})$ ósamfellt.


\subsection{}
Gerið grein fyrir því hvernig punktuppspretturnar eru meðhöndlaðar í forritinu. Sýnið með einni keyrslu hvaða áhrif það hefur á lausn ef punktuppsprettu er bætt við hægri hliðina. 

\par
\textbf{Lausn:} \par
Við höfum gert ráð fyrir að hægri hliðin $f$ í afleiðujöfnunni sé samfellt. Getum líka bætt við það liðum sem eru margfeldi fasta og Dirac-delta-falls. Slíkir liðir lýsa orkuuppsprettum í stökum punktum.  Dirac-fallið í punktinum $r$ táknum við með $\delta _{r}$ . Eina reiknireglan sem við þurfum að kunna er að 

\begin{center}$\int_{c}^{d}\delta _{r}(x) \varphi (x)dx=\varphi(r)$\end{center}

fyrir sérhvert fall $\varphi$ á grennd um lokaða bilið $[c,d]$ og sérhvern punt $r\in[c,d]$. Af þessu leiðir að ef $r=x_{k}$ er einn af punktunum í skipitingu á bilinu [a,b] og $\varphi_{j}$ er þúfugrunnfallið númer $j$, þá er 

\begin{center}$\int_{\alpha}^{\beta}\delta_{x_{k}}(x)\varphi_{j}(x)dx = \varphi_{j}(x_{k})$
=
$\begin{Bmatrix}
1, j=k,
\\ 
0, j \neq k .
\end{Bmatrix}$\end{center}

Í forritinu bætum við því punktuppsprettunum við hægri hlið vigursinns b, með eftirfarandi kóða: 

\lstinputlisting{GalerkinadferdEndir1_3.m}
 


\subsection{}
Finnið sjálf heppilegt sýnidæmi til þess að prófa hvort forritið virki réttt. Kannið hvort skekkjan í nálguninni er $O(h^2)$, þar sem $h$ táknar billengd í jafnri skiptingu með því að prófa forritið í sýnidæminu með fjölda bila $N$=2,4,8....

\par
\textbf{Lausn:}\par 
Heppilegt sýnidæmi sem við fundum er í heimadæmum 8.1. Byrjum á því að skilgreina breyturnar. 
\begin{verbatim}
p = @(x) x;
q = @(x) 1./x;
f = @(x) -2 + 0.*x;
u=@(x) x.*log(x)-x;
x = [1:1/3:2];
alpha=[0,1];
beta = [-1,-2];
gamma=[0,-2];
r=2;
Q=0;
\end{verbatim}
Köllum svo á fallið. 
\begin{verbatim}

>> galerkinadferd_2(@(x)x,@(x)(1./x),@(x)-2+0.*x,[1:1/3:2],[0,1],[-1,-2],[0,-2],2,0)

ans =

   -0.9977
   -0.9424
   -0.8039
   -0.5987
\end{verbatim}
 Plottum svo upp. 
 \begin{verbatim}
 >> plot(linspace(1,2,4),ans)
 >> hold on
 >> plot(linspace(1,2,100),u(linspace(1,2,100)),'r')
 \end{verbatim}
 
 Fáum þá út fallið og nálgunarfallið. 
 
  \begin{figure}[h!]
      \centering
      \includegraphics[width=0.9\textwidth]{nalgun1_4.png}
      \caption{Mynd af fallinu og nálguninni.}
      \label{fig:awesome_image1}
  \end{figure}
  
  Sjáum hér að nálguninn er góð enn ef að við fjölgum bilunum þá verður hún ennþá meira nákvæm.  Sjáum hér að neðan mynd þar sem bilin eru orðin 1000 og þá eru línurnar tvær svo gott sem samliggjandi. 
  
   \begin{figure}[h!]
        \centering
        \includegraphics[width=0.9\textwidth]{nalgun2_4.png}
        \caption{Mynd af fallinu og nálguninni.}
        \label{fig:awesome_image2}
    \end{figure}
\section{Straujárnið hennar Mömmu}



\subsection{}

Skrifið $\verb|matlab|$-forrit sem notar forritið ykkar$ \verb|Galerkinadferd| $ úr síðasta lið til þess að finna nálgun á lausn þessa verkefnis. Notið fjögurra punkta skiptingu  $x_{0}=a, x_{1}=s, x_{2}=r$ og $x_{3}=b$.  Framkvæmið fastapunktsítrekunina þar til endurbítin í ítrekuninni er innan við 1\%.  Markmiðið er að þið prófið ykkur áfram með að stilla gildið á Q þannig að út komi hitastigið $u(b)\approx 200°C$. (Hámarksafllið í venulegu straujárni er 1200-1500W.) Þegar þetta gildi á Q er fundið eigði þi ða teikna upp hitastigið sem fall af $x$. 

\par
\textbf{Lausn:}\par
Byrjum á því að skilgreina allar breyturnar.
\begin{verbatim}
h1 = 5;
h2 = 50;
e = 0.2;
sigma = 5.669*10^-8;
T0 = 298;
T1 = 298;
l1 = 0.1;
l2 = 140;
gamma1 = h1*T0;
a = 0;
s = 0.008;
r = 0.01;
b = 0.016;
x =[0, 0.0008, 0.01, 0.016];
q = @(x) x.*0;
p = @(x) l1 + (l2 - l1)*(x >= r); %þetta er heavisidelidur
f = @(x) x.*0;
beta = [0.1,140]; %l1,l2
Q = 235/0.024;
Q_0 = Q/0.024;
v=[473];
skekkja = 1;
\end{verbatim}
 Nú keyrum við eftirfarandi forrit og fáum út hitastig. 
 
 \begin{verbatim}
 while skekkja>0.01
     %itra galerkin til að haekka hk
     hk=h2+e*sigma*(v(end) +T1)*(v(end)^2+T1^2);
     gamma=[h1*T0,hk*T1];
     alpha=[h1,hk];
     v1=v(end);
     v = galerkinadferd_2( p,q,f,x,alpha,beta,gamma,r,Q );
     skekkja=abs(v1-v(end))/v1;
 end
 disp(v(end))
 plot(x,(v-273))
 
   472.6239
 
 \end{verbatim}
 Sjáum að hitastigið sem kemur út er í kelvingráðunm og því drögum við frá 273 til að fá út hitastigið í celsíus $472.6239-273=199.6239$ og við tökum okkur það bessaleyfi að námunda það upp í 200°C og svo plottum við einnig mynd þar sem við sjáum hitastigið á grafi. 
 \begin{figure}[h!]
      \centering
      \includegraphics[width=0.9\textwidth]{2_2_e_0_2.png}
      \caption{Hitastigsmynd af straujárninu hennar Mömmu.}
      \label{fig:awesome_image3}
  \end{figure}
  
  Sjáum á ferlinu að hitastigið hækkar hratt upp og svo helst það í sirka 200°C.
\subsection{}
Leysið verkefnið aftur með gildið á $Q$ sem þið fundið, en sleppið geislunarliðnum.  ÞEtta þýðir að þið eigið að setja $\epsilon = 0$. Hver er munurinn á lausnunum? Er nauðsynlegt að taka tillit til geislunarliðarins.  

\par
\textbf{Lausn:}\par

við keyrum  nákvæmlega sama í gegn í Matlab nema við breytum einni breytu þannig að \verb|e=0| þá sjáum við að myndin breytist og hitastigið skekkist um sirka 8°C

\begin{verbatim}
481.5692

>> 481.5692-273

ans =

  208.5692
\end{verbatim}

Nú skulum við skoða myndina. 
 \begin{figure}[h!]
      \centering
      \includegraphics[width=0.9\textwidth]{2_2_e_0.png}
      \caption{Hitastigsmynd af straujárninu hennar Mömmu.}
      \label{fig:awesome_image4}
  \end{figure}
  SJáum að hún er alveg eins nema bara með öðruvísi hitastig. 


\subsection{}
Útskýrið hvað gerist ef tekin er fínni skiptinn

\par
\textbf{Lausn:}\par

Ef tekin er fínni skipting þá koma gildi sem eru einfaldlega nær 200°C.  


\section{Dirichlet verkefnið á rétthyrningi}

\subsection{}

Útfæra Dirichletverkfni í  \verb|matlab|. 

\par
\textbf{Lausn:}
\par
\begin{verbatim}
function [W,A,RHS]=Dirichletverkefni(a,b,c,d,p,q,f,gamma,P,Q,R,U,h)
%Author: Andri Freyr Thorgeirsson, Erna Gudrun Thorsteinsdottir og
%        Rikhardur Thor Rognvaldsson

% Breytur

% Setja upp reikninetid
M=(b-a)/h+1;               % Fjoldi hnitpunkta i x att
N=(d-c)/h+1;            % Fjoldi hnitpunkta i y att
% Skilgreinum staerd fylkis og haegri handar
A=zeros(M*N,M*N);       % Fylkid okkar er jafn stort og fjoldi punkta
RHS=zeros(M*N,1);       % Haegri hlid hefur somu staerd og fylkid en er vigur
ki = 1;                 % Dummy breyta fyrir visun í Rs
kj = 1;                 % Dummy breyta fyrir vísun í Qs
%Setjum gildi i fylki og haegri hlid
%Byrjum a því að finna nalgunargildi á R og Q
x_vigur = linspace(a,b,M);
y_vigur = linspace(c,d,N);
r_hnit=NaN;
p_hnit=NaN;
if ~isempty(R)
    r_hnit = zeros(length(R(1,:)),length(R(:,1)));
    for i = 1:length(R(:,1))
        r_hnit(i,1) = min(find(x_vigur>=R(i,1)));
        r_hnit(i,2) = min(find(y_vigur>=R(i,2)));
    end
end
if ~isempty(P)
    
    p_hnit = zeros(length(P(1,:)),length(P(:,1)));
    for j = 1:length(R(:,1))
        p_hnit(j,1) = min(find(x_vigur>=P(j,1)));
        p_hnit(j,2) = min(find(y_vigur>=P(j,2)));
    end
end



for n=1:N
    for m=1:M
        
        % Vid kjosum ad rada T_m,n thar sem fyrst er n=1 og m haekkar fra 1 til M,
        % Thvi naest haekkum vid n i 2 og haekkum m fra 1 til M, haekkum tha n i 3 og
        % haekkum m fra 1 til M og svo framvegis thar til m is jafnt M
        
        s=m+M*(n-1);               % Numer linu fylkis sem samsvarar m,n
        s_m_plus1=m+1+M*(n-1);     % Numer linu fylkis sem samsvarar m+1,n
        s_m_minus1=m-1+M*(n-1);    % Numer linu fylkis sem samsvarar m-1,n
        s_n_plus1=m+M*(n+1-1);     % Numer linu fylkis sem samsvarar m,n+1
        s_n_minus1=m+M*(n-1-1);    % Numer linu fylkis sem samsvarar m,n-1
        x = a+(m-1)*h;
        y = c+(n-1)*h;
        x_half = a+(m-1)*(h/2);
        y_half = c+(n-1)*(h/2);
        % Viðbót 2 - fast gildi í ákveðnum punkti
        
        % Jaðarskilyrði - u = gamma
        % Skilyrdi: (n==1) -> Viljum skoda nedstu linuna kassans
        % Skilyrdi: (n==N) -> Viljum skoda efstu linuna kassans
        % Skilyrdi: (m==1) -> Viljum skoda vinstri hlid kassans
        % Skilyrdi: (m==M) -> Viljum skoda haegri hlid kassans
        if  (m==1) || (M==m) || (n==1) || (N==n)
            A(s,s) = 1;
            RHS(s) = gamma(x,y);
            fprintf('Jadarpunktur n=%.0f , m = %.0f\n',n,m)
        end
        
        % Innri hnutpunktar kassans - jafna 22.3
        % Skilyrdi: (n>1)  -> Viljum ekki skoda nedstu linuna
        % Skilyrdi: (m>n) -> Viljum lenda haegra meginn vid vinstri
        %                    hlidarlinu thrihyrningsins
        % Skilyrdi: (m<M-n+1) -> Viljum lenda vinstra meginn vid haegri
        %                        hlidarlinu thrihyrningsins
        % Skilyrdi: (n<N) -> Viljum ekki lenda i topppunktinum.
        if (n>1) && (m>1) && (m<M) && (n<N)
            A(s,s) = ((p(x_half+h,y)+p(x_half-h,y)+p(x,y_half-h)+p(x,y_half+h))/h^2)+q(x,y);
            A(s,s_m_plus1) = -p(x_half+h,y)/h^2;   %p_j,r
            A(s,s_m_minus1) = -p(x_half-h,y)/h^2;  %p_j,l
            A(s,s_n_plus1) = -p(x,y_half+h)/h^2;   %p_j,t
            A(s,s_n_minus1) = -p(x,y_half-h)/h^2;  %p_j,s
            RHS(s) = f(x,y);
            fprintf('Innripunktur n=%.0f , m = %.0f\n',n,m)
            if ~isnan(p_hnit)
                if m == p_hnit(kj,1) && p_hnit(kj,2) == n && kj<=length(P(:,1))
                    RHS(s) = f(x,y)+sum(Q(kj)/h^2);
                    kj=kj+1;
                    fprintf('Sérstöðupunktur - Punktuppspretta x=%.2f , y = %.2f , n=%.0f , m = %.0f\n',x,y,n,m)
                end
            end
        end
        if  ki<=length(R(:,1))
            if m == r_hnit(ki,1) && r_hnit(ki,2) == n 
                A(s,:) = 0;
                A(s,s) = 1;
                RHS(s) = U(ki);
                ki = ki+1;
                fprintf('Sérstöðupunktur - FAST GILDI x=%.2f , y = %.2f , n=%.2f , m = %.0f\n',x,y,n,m)
                disp(A(s,:))
            end
        end
    end
end
%Finnum lausn
A_sparse=sparse(A);
W_list=A_sparse\RHS;   % Thetta er vigur med hitastigum rodudum eins og vid til-
% greindum ad ofan
T_grid=zeros(N,M);
for n=1:N
    for m=1:M
        s=m+M*(n-1);   % Numer linu fylkis sem samsvarar m,n
        T_grid(n,m) = W_list(s); % Setjum inn gildi sem samsvarar
        % gildi fyrir thrihyrninginn okkar.
    end
end
W=T_grid;

\end{verbatim}
\subsection{}
Útskýrið hvers vegna uppsprettuliðirnnir eru meðhöndlaðir með því að setja $b_{j} = f(x_{j},y_{k})+h^{-2}$ ef $(x_{j},y_{j})$ er sá punktur netsins sem næstur er $P_s$.  

\par
\textbf{Lausn:}\par
Við erum með net með uppsprettupunkti.  En til þess að geta fundið hann þá þurfum við að vita fyrirframstaðsetingu á uppsprettupunktinum. 
Það gerum við með eftirfarandi.
við vitum að  
\begin{center}$-\nabla(p \nabla u) + qu= f +Q \delta_{Ps}$ \end{center}
ef uppsprettupunkturinn $\delta_{Ps}$ er í aðeins einum fyrirfram þekktum punkti þá vitum við að 
\begin{center}$\int \int \delta_{Ps}(x,y)dA=1$ \end{center}
 svo ef við heildum yfir jaðarinn í kringum punktuppsprettuna þá fá um við út 
 $.....+q_i*h^2=f_i*h^2+Q_s$
 Síðan stöðlum við þessa jöfnu með því að deila öllu með $h^2$
 og þá fáum við út 
 \begin{center}$b_i= f_i+\frac{Q_s]}{h^2}$\end{center}
\subsection{}
Þið þurfið að sannfæra ykkur um forritið reikni rétt.  Tökum t.d. fallið $u(x,y)=x^2-y^2$ veljið D=[-1,1]x[-1,1], og $\gamma(x,y)=x^2-y^2$ á öllum jaðrinum, þá eigið þið að fá nálgunarfall sem er nánast eins og rétta lausnin.

\par
\textbf{Lausn:}\par
byrjum á því að skilgreina breyturnar okkar. 
\begin{verbatim}
a=-1;
b=1;
c=-1;
d=1;
gamma= @(x,y) x.^2-y.^2;
p=@(x,y) 1;
q=@(x,y) 0;
f=@(x,y) 0;
h= 0.1;
\end{verbatim}

Köllum svo á fallið:
\begin{verbatim}
>> [W,A,RHS]=Dirichletverkefni(a,b,c,d,p,q,f,gamma,[1,1],[0],[1,0],[1],0.1)
>> surf(linspace(a,b,length(W(:,1))),linspace(c,d,length(W(1,:))),W'); 
\end{verbatim}
 \begin{figure}[h!]
     \centering
     \includegraphics[width=0.9\textwidth]{sodulmynd1.png}
     \caption{Söðulmynd}
     \label{fig:awesome_image5}
 \end{figure}
 \newpage
þessi liður gefur okkur svo contour af fallinu. 
\begin{verbatim}
>> contour(linspace(a,b,length(W(:,1))),linspace(c,d,length(W(1,:))),W');
\end{verbatim}
 \begin{figure}[h!]
     \centering
     \includegraphics[width=0.9\textwidth]{sodulmynd_contour.png}
     \caption{Contour-mynd af Söðulmynd}
     \label{fig:awesome_image6}
 \end{figure}
\newpage


\subsection{}
Notið forritið ykkar til að þess að taka svæðið á D]-1,1]x[-1,1[, og reikna út og teikna upp lausinina á 


\begin{center}
$
\left\{\begin{matrix}
$$\nabla^2u=0
\\ 
\alpha=1, \beta=0, \gamma(x,y)$$
\end{matrix}\right.
$
= 
$
 \left\{\begin{matrix}
  5max
 \begin{Bmatrix}
 $$cos(\pi y),0\end{Bmatrix}
 , x=\pm 1,\
 \\5max
 \begin{Bmatrix}
 cos(\pi y),0
 \end{Bmatrix}, y= \pm 1. 
 $$
 
 \end{matrix}\right.
 $
\end{center}
 
 Setjið teikningu af grafi lausnarinnar í skýrsluna. 
 
 \par
 \textbf{Lausn:}\par
 
 Byrjum á því að skilgreina breyturnar.
 \begin{verbatim}
 a=-1;
 b=1;
 c=-1;
 d=1;
 gamma= @(x,y) x.^2-y.^2;
 p=@(x,y) 1;
 q=@(x,y) 0;
 f=@(x,y) 0;
 h= 0.1;
 \end{verbatim}
 Svo köllum við á fallið:
 \begin{verbatim}
 >>[W,A,RHS]=Dirichletverkefni(a,b,c,d,p,q,f,gamma,[],[],[],[],0.1)
 >> surf(linspace(a,b,length(W(:,1))),linspace(c,d,length(W(1,:))),W');
 \end{verbatim}
Sjáum að myndin líkist kirkjunni í Kópavogi.
  \begin{figure}[h!]
      \centering
      \includegraphics[width=0.9\textwidth]{gudshusid.png}
      \caption{Guðshúsið í Kópavogi}
      \label{fig:awesome_image7}
  \end{figure} 
  
  
 Köllum svo aftur á fallið til að fá contour-mynd.
 \begin{verbatim}
 >> contour(linspace(a,b,length(W(:,1))),linspace(c,d,length(W(1,:))),W');
 \end{verbatim}
 
 Fáum þá út myndina.
 
 \begin{figure}[h!]
       \centering
       \includegraphics[width=0.9\textwidth]{gudshusid_contour.png}
       \caption{Countour-mynd af Guðshúsinu í Kópavogi}
       \label{fig:awesome_image8}
   \end{figure}
   
\newpage
 
 \subsection{}
 Líkið eftir Hróarskeldutjaldinu. Útskýrið hvaða formúlur þið notað og setjið myndí skýrsluna. 
 
 \par
 \textbf{Lausn:}\par
 
 Byrjum á því að skilgreina eftirfarandi breytur:
 \begin{verbatim}
 a=-2;
 b=2;
 c=-0.5;
 d=2;
 gamma=@(x,y) hroarskelda_gamma(x,y);
 p = @(x,y) 1+0.*x+0.*y;
 q = @(x,y) 0+0.*x+0.*y;
 f = @(x,y) 0+0.*x+0.*y;
 h=0.1;
 \end{verbatim}
 Aukaforritið   \verb|hroarskelda_gamma.m| er hægt að finna í viðauka hér að neðan.  Við köllum svona á forritið til að fá eftirfarandi mynd:
 \begin{verbatim}
 >>[W,A,RHS]=Dirichletverkefni(a,b,c,d,p,q,f,gamma,[],[0],[0,0;-0.9,1;0.9,1],[1,1,1],0.1);
 >> surf(linspace(a,b,length(W(:,1))),linspace(c,d,length(W(1,:))),W');
 
 \end{verbatim}
 Sjáum hér að ofan hvað  \verb|a,b,c,d,p,q,f,gamma| er skilgreint sem. 
 
 Við völdum að þar sem punktuppspretturnar myndu byrja væru hnitin  \verb|[0,0],[-0.9,1],[0.9,1]|  síðan hvað "magnitude-ið" væri á þessum punktuppsprettum en allar höfðu þær styrkinn 1. Út kom þá eftirfarandi myndir. 
 
 \begin{figure}[h!]
     \centering
     \includegraphics[width=0.9\textwidth]{hroarskelda_hlid.png}
     \caption{Hróarskeldutjaldið á hlið}
     \label{fig:awesome_image9}
 \end{figure}
 
 \begin{figure}[h!]
     \centering
     \includegraphics[width=0.9\textwidth]{hroarskelda_top.png}
     \caption{Hróarskeldutjaldið ofan frá }
     \label{fig:awesome_image10}
 \end{figure}
 
  \begin{figure}[h!]
      \centering
      \includegraphics[width=0.9\textwidth]{hroarskelda_party.png}
      \caption{Hróarskeldutjaldið í allri sinni dýrð.}
      \label{fig:awesome_image11}
  \end{figure}
  \newpage
  
  
 \subsection{}
 Hannið nýtt tjald til þess að setja upp við Arnarhól á menningarnótt í Reykjavík. Útskýrið hönnunina og setjið mynd í skýrsluna.
 
 \par
 \textbf{Lausn:}\par
 
 
 
\section{Lágmarksfletir}

\subsection{}
$(x,y)= \sqrt{\cosh (x)^{2}-y^{2}}$  uppfyllir hlutafleiðujöfnuna fyrir lágmarksflöt. Notið það á rétthyrningnum D = [-1,1] * [-1,1] til þess að kanna hvort forritið reiknar.

\par
\textbf{Lausn:}\par
Byrjum á því að skilgreina allar breytur. 
\begin{verbatim}
a=0;
b=1;
c=-1;
d=1;
gamma = @(x,y) sqrt(cosh(x).^2-y.^2);
p = @(x,y) -1+0.*x+0.*y;
q = @(x,y) 0+0.*x+0.*y;
f = @(x,y) 0+0.*x+0.*y;
[W_dummy,A,RSH] = Dirichletverkefni(a,b,c,d,p,q,f,gamma,[],[],[],[],0.02);
W0 = Dirichletverkefni(a,b,c,d,p,q,f,gamma,[],[],[],[],0.02);
k=1;
eps=0.01;
kmax = 5;
error = 2*eps;
\end{verbatim}

Því næst keyrum við eftirfarandi forrit í matlab \verb|lagmarksflotur.m| er hægt að finna í viðauka.  
\begin{verbatim}
while(error > eps && k < kmax)
   W1=lagmarksflotur(a,b,c,d,p,q,f,gamma,[],[],[],[],0.02,W0');
   error = max(max(abs(W1-W0)));
   W0 = W1;
   k=k+1;
end

>> surf(linspace(a,b,length(W1(:,1))),linspace(c,d,length(W1(1,:))),W1');
\end{verbatim}

næst plottum við myndina. 
  \begin{figure}[h!]
      \centering
      \includegraphics[width=0.9\textwidth]{lagmarksflotur_hlid.png}
      \caption{Hliðarmynd af fletinum.}
      \label{fig:awesome_image12}
  \end{figure}
  
    \begin{figure}[h!]
        \centering
        \includegraphics[width=0.9\textwidth]{lagmarksflotur_framan.png}
        \caption{Framan á flötinn.}
        \label{fig:awesome_image12}
    \end{figure}

  \begin{figure}[h!]
      \centering
      \includegraphics[width=0.9\textwidth]{lagmarksflotur.png}
      \caption{Flöturinn}
      \label{fig:awesome_image13}
  \end{figure}
  Köllum á contour-mynd af fallinu.
  \begin{verbatim}
  >> contour(linspace(a,b,length(W1(:,1))),linspace(c,d,length(W1(1,:))),W1');
  \end{verbatim}

    \begin{figure}[h!]
        \centering
        \includegraphics[width=0.9\textwidth]{lagmarksflotur_contour.png}
        \caption{Contour-mynd af fletinum.}
        \label{fig:awesome_image13}
    \end{figure}
\newpage

\subsection{}
Gerið grein fyrir stærð netsins sem þið notið, hvaða þolmörk þið notið til þess að stöðva ítrekunina og hversu langan tíma keyrslan tók á tölvunni ykkar.

 \par
 \textbf{Lausn:}\par
Stærð netsins hjá okkur er W1=101x101.  Þolmörkin sjáum við í eftirfarandi kóða
\begin{verbatim}
while(error > eps && k < kmax)
   W1=lagmarksflotur(a,b,c,d,p,q,f,gamma,[],[],[],[],0.02,W0');
   error = max(max(abs(W1-W0)));
   W0 = W1;
   k=k+1;
end
\end{verbatim} 
Sjáum að við keyrum þetta 5x í gegn. Notum \verb|tic toc| skipunina til að finna út nákvæma tímasetningu og hún er 6.787692 sekúndur.

\subsection{}
Berið saman á mynd hver munurinn er á því sem fékkst út í upphafsskrefinu og lokaskrefinu. Þetta má einfaldlega gera með því að teikna upp mismuninn.

\subsection{}
Keyrið forritið með dæmið ykkar um Hróarskeldutjaldið eða Arnarhólstjaldið. Hver er munurinná lausninni á jöfnunni fyrir lágmarksflöt miðað við graf lausnarinnar á Laplace-jöfnu meðsömu jaðarskilyrðum. Setjið fallega mynd á forsíðuna.
 
 
\newpage
\begin{flushright}
 \today
\end{flushright}
\end{document}
